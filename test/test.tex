%&LaTeX
\documentclass{article}
\usepackage{multicol}
\usepackage{hyperref}
\def\R2Lurl#1#2{\mbox{\href{#1}{\tt #2}}}               
                                                                                                    
                                                                                                    
                                                                                                    
                                                                                                    
                                                                                                    
                                                                                                    
                                                                                                    
                                                                                                    
                                                                                                    
\newcommand{\tab}{\hspace{5mm}}


\begin{document}



\section*{Text Style}
{\underline {This line is underlined.}}\\
\textbf{This is in bold style.}\\
\textit{This is in italic style.}\\
\textbf{\textit{This is in italic and bold style.}}\\
One letter in this line is \textit{i}talic, one is \textbf{b}old.\\
This line has a superscript$^{1}$ in it.\\
This line has a subscript$_{2}$ in it.\\
This line has both a subscript and a superscript$^{(1)}$$_{(}$$_{\mathbf{2}}$$_{)}$.\\
How about \ensuremath{\alpha}$_{\ensuremath{\lambda}}$ and b$_{1}$?\\
\textbf{\textit{(Partly italic)}} \textbf{and \{completely bold\}.}


\textsc{This is a text in small caps.}

{\scriptsize Tiny (6),} {\footnotesize script size (7),} {\small footnote size (9),} small 
(10), normal (12), {\large large (14),} {\Large Large (16),} {\LARGE LARGE (18),} {\huge huge 
(24),} {\Huge Huge (36).}



\section*{Paragraph}

\begin{flushright}
This is right justified.


\end{flushright}

\begin{center}
This is centered text.


\end{center}

To see paragraph indentation, switch off the \textit{ignoreRuler} 
option in the r2l-pref file.\\
This text is indented with the ruler at 1 cm of left and 6 cm 
of right paper borders. This text is indented with the ruler at 
1 cm of left and 6 cm of right paper borders. This text is indented with 
the ruler at 1 cm of left and 6 cm of right paper borders. This 
text is indented with the ruler at 1 cm of left and 6 cm of right 
paper borders. This text is indented with the ruler at 1 cm of 
left and 6 cm of right paper borders. This text is indented with 
the ruler at 1 cm of left and 6 cm of right paper borders. This 
text is indented with the ruler at 1 cm of left and 6 cm of right 
paper borders. This text is indented with the ruler at 1 cm of 
left and 6 cm of right paper borders. This text is indented with 
the ruler at 1 cm of left and 6 cm of right paper borders.




\begin{multicols}{2}
This is a text which is in a two-column environment of WinWord. 
This is a text which is in a two-column environment of WinWord. 
This is a text which is in a two-column environment of WinWord. 
This is a text which is in a two-column environment of WinWord. 
This is a text which is in a two-column environment of WinWord. 
This is a text which is in a two-column environment of WinWord. 
This is a text which is in a two-column environment of WinWord. 
This is a text which is in a two-column environment of WinWord. 
This is a text which is in a two-column environment of WinWord. 
This is a text which is in a two-column environment of WinWord. 
This is a text which is in a two-column environment of WinWord. 
This is a text which is in a two-column environment of WinWord. 
This is a text which is in a two-column environment of WinWord. 
This is a text which is in a two-column environment of WinWord. 
This is a text which is in a two-column environment of WinWord. 
This is a text which is in a two-column environment of WinWord. 
This is a text which is in a two-column environment of WinWord. 
This is a text which is in a two-column environment of WinWord. 
This is a text which is in a two-column environ -column environment 
of WinWord. This is a text which is in a two-column environment 
of WinWord. This is a text which is in a two-column environment 
of WinWord. This is a text which is in a two-column environment.



\section*{
}

\end{multicols}


\section*{Color}
To see colors, switch off the \textit{ignoreColor} option in the r2l-pref 
file.\\
This is in red.\\
Words in different colors.



\section*{Symbols}
\ensuremath{\rightarrow}\ensuremath{\leftarrow}\ensuremath{\clubsuit}\ensuremath{\diamondsuit}\ensuremath{\heartsuit}{\texttrademark}{\copyright}\ensuremath{\supseteq}\ensuremath{\subseteq}\ensuremath{\pm}\ensuremath{\sqrt{}}\ensuremath{\Rightarrow}\ensuremath{\Leftrightarrow}{\copyright}



\section*{Greek Letters}
\ensuremath{A}\ensuremath{B}\ensuremath{X}\ensuremath{\Delta}\ensuremath{E}\ensuremath{\Phi}\ensuremath{\Gamma}\ensuremath{H}\ensuremath{I}\ensuremath{\vartheta}\ensuremath{K}\ensuremath{\Lambda}\ensuremath{M}\ensuremath{N}\ensuremath{O}\ensuremath{\Pi}\ensuremath{\Theta}\ensuremath{P}\ensuremath{\Sigma}\ensuremath{T}Y\ensuremath{\varsigma}\ensuremath{\Omega}\ensuremath{\Xi}\ensuremath{\Psi}\ensuremath{Z}\\
\ensuremath{\alpha}\ensuremath{\beta}\ensuremath{\chi}\ensuremath{\delta}\ensuremath{\epsilon}\ensuremath{\phi}\ensuremath{\gamma}\ensuremath{\eta}\ensuremath{\iota}\ensuremath{\varphi}\ensuremath{\kappa}\ensuremath{\lambda}\ensuremath{\mu}\ensuremath{\nu}\ensuremath{o}\ensuremath{\pi}\ensuremath{\theta}\ensuremath{\rho}\ensuremath{\sigma}\ensuremath{\tau}\ensuremath{\upsilon}\ensuremath{\varpi}\ensuremath{\omega}\ensuremath{\xi}\ensuremath{\psi}\ensuremath{\zeta}



\section*{Footnote}
This file contains a footnote\footnote{Footnote text\dots  \\
This on a separate line.}.



\section*{Hyperlink}
Please visit rtf2latex2e's \R2Lurl{http://members.home.net/setlur/rtf2latex2e }{website}.

\end{document}
