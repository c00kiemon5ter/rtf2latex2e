\documentclass{article}
\usepackage{color}
\usepackage{hyperref}

\definecolor{color2}{rgb}{0.000,0.000,1.000}
\definecolor{color4}{rgb}{0.000,1.000,0.000}
\definecolor{color6}{rgb}{1.000,0.000,0.000}
\definecolor{color10}{rgb}{0.000,0.502,0.502}

\def\rtf2latex2e{{\it rtf2latex2e}}

\title{\rtf2latex2e Documentation\\
version 2.0.1} 
\author{Ujwal S. Sathyam\\ Scott Prahl}
\date{09 Feb 2011}

\begin{document} 
\maketitle

\section{Introduction 
\label{intro}} \rtf2latex2e is an RTF$\rightarrow${\LaTeX}
converter that takes as its input RTF files produced by Microsoft Word
and comparable word processors such as Star Office and generates a
\TeX-able ``{\it .tex}'' file.  It has the capability to handle fairly
complex RTF files containing figures, tables, and equations.  
\rtf2latex2e is written using standard $C$ and
should compile on any platform supporting a $C$ compiler.  It has
been tested on Mac OS X, Linux (Intel), and Windows.

\rtf2latex2e uses the generic
\href{http://www.snake.net/software/RTF/}{RTF reader}
framework by Paul DuBois.  The framework is a general purpose tool for
processing RTF files and may be configured in a well-defined manner to
allow it to be used with a variety of writers generating different
output formats.  This provides a method for generating RTF-to-XXX
translators.  Paul seems to have stopped developing the
reader code, therefore Ujwal Sathyam (in 1999) had to adapt it quite a bit to handle the
latest version of RTF.  More code evolution took place during
development of version 2.0.

If you expect a WYSIWYG reproduction of your RTF file, you may be
disappointed.  Our main concern has been translating the essential
features of the RTF file such as characters, figures, tables, and
equations.  Visual formatting
such as ruler positions, tabs (until I figure out a good way of doing
this), paragraph indentations, and other fluff has been largely ignored.  
The translated {\LaTeXe} file will require some manual editing to put the
finishing touches.  We just want to make that task a little easier.  In
our opinion, expecting a WYSIWYG reproduction is not practical and
misses the point entirely.

\section{Installation}
The source code is available at \href{http://sourceforge.net/projects/rtf2latex2e/}{SourceForge}
as well as compiled binaries for Windows systems. 
Users of MacOSX, Unix, and linux systems will have to compile the source code,
which is a fairly straightforward process.

\subsection{Unix \label{Unix_install}}
Change to the \textit{rtf2latex2e} directory and type:
\begin{verbatim}
   prompt> make
\end{verbatim}
This will compile the sources and create a binary called
\textbf{rtf2latex2e}.
You can choose to install the binary in a convenient location.
If you want to install into the default directory
\textit{/usr/local/bin},
Installation is done by:
\begin{verbatim}
   prompt> sudo make install
\end{verbatim}
otherwise edit the \texttt{\$PREFIX} variable in \textit{Makefile}
to install the files where you would like. 

\subsection{Mac OS X}
Just follow the above direction, but an added feature is available.
Since so many Mac RTF files come with embedded \textit{.pict} images, 
it is possible to include automatic translation of \textit{.pict}$\rightarrow$\textit{.pdf}
in the binary by uncommenting (removing the \#) these lines in the Makefile before
compiling
\begin{verbatim}
   #CFLAGS  :=$(CFLAGS) -m32 -DPICT2PDF
   #LDFLAGS :=$(LDFLAGS) -m32 -framework ApplicationServices
\end{verbatim}
On a Mac, this will causes {\rtf2latex2e} to automatically convert the PICT images to PDF images
and change the \verb#\includegraphics# file names.\footnote{It is understandable that 
has deprecated the PICT format and is pushing PDF.  PICT is an old format, true,
but I am always stumbling across legacy files with PICT images.  The only way to properly
convert the PICT vector image to a PDF vector images is to use the \texttt{QDPictCreateWithURL()}
call.  This function is not available when compiling 64-bit binaries and therefore the \texttt{-m32}
option is required while compiling and linking. }  Obviously the proper framework is
not available on other systems so this option is commented out by default.

\subsection{Windows}
Windows users get a pre-compiled binary rtf2latex2e.exe to be run from
the command prompt. Just run it with the RTF file to be converted as the argument.
To compile, uncomment the following line in the Makefile
\begin{verbatim}
    PLATFORM?=-DMSWIN # Windows
\end{verbatim}

\section{Using \rtf2latex2e}

This is a command-line program, which may be run with a variety of options
\begin{verbatim}
    prompt> rtf2latex2e [options] file.rtf
    
    prompt> rtf2latex2e foo              convert foo.rtf to foo.tex
    prompt> rtf2latex2e -p 33 -t 4 foo   minimal latex mark-up
    prompt> rtf2latex2e -e 15 foo        debug failed eqn conversion
    prompt> rtf2latex2e foo-eqn003.eqn   debug third equation 
    prompt> rtf2latex2e -D foo           put files in foo-latex dir
    prompt> rtf2latex2e foo.rtfd         convert to foo.rtfd/TXT.tex
\end{verbatim}

\subsection{The trivial options}
The trivial options are
\begin{verbatim}
  -h               display help
  -v               version information
\end{verbatim}
and should require no explanation, so I won't bother.

\subsection{The new directory option}
The option to create a new directory is interesting and useful
\begin{verbatim}
  -D               make a new directory for latex and extracted images
\end{verbatim}
because RTF files often contain dozens of image files.  All these
files are extracted during conversion and this will clutter your
current working directory.  The \texttt{-D} option avoids this and create a single
location for all the new latex files by locating all of them in a new
directory.  For example
\begin{verbatim}
    prompt> rtf2latex2e -D file.rtf
\end{verbatim}
creates a new directory called \textit{file-latex} and places the converted
latex file \textit{file.tex} inside.  The file names are adjusted so that
you can \texttt{cd} to that directory and type 
\begin{verbatim}
    prompt> cd file-latex
    prompt> pdflatex file.tex
\end{verbatim}
to create a PDF file, for example.

\subsection{The preference path option}
This is important.  
\begin{verbatim}
  -P path/to/preferred/preference/directory
\end{verbatim}
Probably more important than it should be. \rtf2latex2e{}
reads a bunch of font encodings \textit{rtf-encoding.*}, \textit{rtf-ctrl}, and
\textit{latex-encoding} before it can convert anything.  These are all located
in the preference directory.\footnote{Arguably, all these files could be compiled
into the binary.  Certainly, the only file that a user might actually modify 
is the \textit{latex-encoding} file, but now that we have UTF8, why bother?} 
In addition to these critical files, some user configurable files \textit{r2l-head},
\textit{r2l-map}, and \textit{r2l-pref} are also read from this location.

There are three ways to specify the location of this directory, 
\begin{itemize}
\item
using the default compiled in location as specified in the \textit{Makefile}.  As
shipped, \rtf2latex2e{} defaults to \textit{/usr/local/share/rtf2latex2e}.
\item
using the shell environment variable \texttt{\$RTFPATH}.  This can be a handy thing
to put in your \textit{.bash\_profile} file if you want to use some other directory
than the default one above.  This option is searched before the above.
\item
using the \texttt{-P path} option.  This gets tried first.  For example, during
development, this option is used extensively because the local \textit{pref}
directory often differs from the system installed version.
\end{itemize}

\subsection{The text conversion option}

Obviously text can have a bunch of formatting applied to it.
{\color{color6} c}{\color{color2} o}{\color{color4} l}{\color{color10}
o}r, {\bf bold}, {\em italic}, {\underline {underlined}}, font selection,
and font size. Basically, the
specific font sizes specified in the RTF file get mapped to the 
the nearest relative latex sizes like \verb#\small# and \verb#\large#.
All these can be turned off with the \texttt{-t} option.  Just add the
numbers together to get a combination of items
\begin{verbatim}
  -t #             text conversion options
      -t1              font size
      -t2              font color
      -t4              font formatting
      -t8              replace tabs with spaces
\end{verbatim}
Thus \texttt{-t 6}          
will cause \rtf2latex2e{} to convert formatting and color, but
ignore size changes and replace all tabs with spaces.

Disabling font size conversion is often a good idea because
\LaTeX{} usually makes reasonable assumptions about the font size.
It is a nuisance if every paragraph is bracketed with 
\verb#{\Large ... }# because the RTF document was written in 
14 point font.

No support for font changes exists in \rtf2latex2e{} version 2.0
--- maybe in another ten years.

\subsection{The paragraph conversion option}

Paragraphs are nasty beasts.  Just consider indenting.  In an RTF
file, indenting can be achieved
\begin{itemize}
\item by using the \verb#\fi# tag to 
indicate the indentation of the first line.  
\item by tabbing to the first tab stop
\item by using spaces
\item by using a style
\end{itemize}
Well, that was not as bad as I remembered it being.  But that is just
the first indent.  There is also line spacing before and after the paragraph,
line spacing, left and right margins, and overall alignment.  
\begin{verbatim}
  -p #             paragraph conversion options
      -p1              'heading 1' style -> '\section{}'
      -p2              indenting
      -p4              space between paragraphs
      -p8              line spacing
      -p16             margins
      -p32             alignment
\end{verbatim}
If all the options are on then a ton of cruft is added to the latex document.
For example, using \texttt{-p 63} each paragraph might come out looking like
\begin{verbatim}
    \vspace{6pt}
    \leftskip=15pt
    \parindent=-15pt
    This architecture allows the reader to remain constant, 
    so that different translators can be built by supplying 
    different writer and driver code....
\end{verbatim}
Although, the \LaTeX{}'ed file looks great, it is a nuisance to
edit.  Since presumeably, this is the only reason that you are 
converting to \LaTeX{}, you probably don't want the extra markup.
I personally like \texttt{-p 33} which is the default for the 
\texttt{-n} option described below.

I should describe the \texttt{-p 1} option, but I really am tired
of this document.  So I won't.

\subsection{The equation conversion option}

The most common source of the RTF files
is Microsoft Word.  Equations in Word are created in Equation Editor
(MathType), and when saved into an RTF file, the equation is saved
as MTEF embedded in an OLE object. \rtf2latex2e{} uses the
cole library\footnote{originally available in 2000
from \url{http://arturo.directmail.org/filtersweb}} to
extract the embedded equations from the OLE structured format.
The equation is then converted into {\LaTeX} format.  If everything
goes well, then the conversion can be surprisingly good.  
If the native equation conversion fails, or if
the option to convert equations is disabled, \rtf2latex2e reads that
picture and outputs the equation as a picture file.

Not surprisingly, sometimes the conversion fails.  One of the major
bug fixes in version 2.0 was to properly extract the MTEF record 
of the equation from the encoded OLE.  During processing, \rtf2latex2e{}
saves this as a separate file.  For example, the thirty-third equation 
when translating \textit{file.rtf} might get saved as \textit{file-eqn033.eqn}.
This file is then processed to produce a string that is inserted into
the converted latex document.  If something goes wrong with processing
this particular equation then it is very handy to be able to access this
equation directly.  So how do you know it was equation 33 and not equation 19?
Use \texttt{-e 15} according to
\begin{verbatim}
  -e #             equation conversion options
      -e1              convert to latex
      -e2              insert image
      -e4              keep intermediate eqn file
      -e8              insert eqn file name in latex document
\end{verbatim}
Then in the converted document (or more likely the partially converted
\textit{file.ltx} document, insert a final \verb#\end{document}# so that
you can see the converted latex file.  The \texttt{-e 2} allows you to
visually compare the converted latex with an image of what it should look
like.  The \texttt{-e 8} option will insert the name of the associated
\textit{.eqn} file.  Then it is just an issue of debugging the equation
using
\begin{verbatim}
    prompt> rtf2latex2e file-eqn033.eqn
\end{verbatim}
and figuring out which one of the MTEF commands is getting mangled.

\subsection{Two useful options}
\begin{verbatim}
  -b               best attempt at matching RTF formatting
  -n               natural latex formatting ... easiest to edit
\end{verbatim}
This is what I do
\begin{verbatim}
   prompt> rtf2latex2e -n test.rtf
\end{verbatim}
which gives the best balance of minimal mark-up with conversion fidelity.

\section{The Preference files}
\subsection{r2l-pref}
\rtf2latex2e reads a preference file \textit{r2l-pref} where you can
specify various options 
\begin{itemize}
\item preambleFirstText
\item preambleSecondText
\item preambleDocClass
\item outputMapFileName
\item pageWidth
\item pageLeft
\item pageRight
\item convertPageSize
\item convertParagraphStyle
\item convertParagraphIndent
\item convertInterParagraphSpace
\item convertParagraphMargin
\item convertParagraphAlignment
\item convertLineSpacing
\item convertTextSize
\item convertTextForm
\item convertTextNoTab
\item convertTextColor
\item convertHypertext
\item convertEquation
\item convertAsDirectory
\item convertTableName
\item convertPict
\end{itemize}
The options are self-explanatory.  If not, then read the comment text in
\textit{r2l-pref}.  These preference selections are overridden by command
line options.

\subsection{latex-encoding}
The default charset for the latex file is Unicode, the text file is written
using UTF8.  This is specified by the file \textit{latex-encoding}.  Others encodings are possible but I think they should all be dropped.  Unicode works with 
\LaTeX{} so why bother with any other representation.  Consider the others
deprecated.

\subsection{r2l-head}
\rtf2latex2e also reads a file (if present) called
\textit{r2l-head}.  In this file, you can specify any additional
packages that you want to use in your {\LaTeX} file, e.g., a babel
hyphenation package or a font encoding.  The contents of this file are
just copied into the preamble of the {\LaTeX} file.  

\subsection{r2l-map}
\rtf2latex2e also reads a file (if present) called
\textit{r2l-map}.  In this file, you can customize mappings for section headings. 
Other style conversion may be possible, but it should be considered
experimental (aka, flakey) at the moment.

\section{Features} 
\rtf2latex2e is designed to convert journal articles, reports, and
letters written in Microsoft Word.  That means I would like it to
handle the following:

\subsection{Figures}

\rtf2latex2e can read figures
of format PICT, EMF, WMF, PNG, and JPEG embedded in RTF files.  These are
the most common formats encountered in RTF files.  When \rtf2latex2e
encounters an embedded figure, it reads out the figure into a separate
file.  The output format of the figure is the same as the format it was
embedded in. On the Mac, \rtf2latex2e will attempt to convert PICT to
PDF files.\footnote{It used to be that all images were converted to EPS
files.  This capability was removed because it other utilities do it
better, I prefer EPS, and \texttt{pdflatex} most formats directly.  It is
better that you convert files as you would like, rather than providing
a poor conversion to EPS.}


\subsection{Tables}
Yeah, it does tables!!  However, this is
the weakest link in the chain and the messiest part of the code.  This
is largely due to the fact that RTF does not have a separate ``Table''
group.  It is also due to the fact that TeX likes to know in advance
the number of columns in the table, and RTF does not tell us that.  Ujwal
spent a lot of time to support tables to this extent.  Some of the
test files have tables in them.  To get an idea of the type of tables
that \rtf2latex2e can handle, take a look at {test/table.rtf}.  
The package \textit{longtable.sty} is used for
generic table handling to take care of tables that span several pages.

Tables have way more mark-up than desired.

\subsection{Character mapping:}

Character mapping is largely
complete for the most common latin scripts.  Characters are translated
by referencing character set maps and the output map file
\textit{latex-encoding}.  The platform and locale dependent character set, e.g. 
latin-2 (Eastern European), is converted to an internal
platform-independent representation by reading the appropriate
character map file, in this case \textit{rtfencoding-cp1250}.  For example, character
192 (hex $0xC1$) represents \textit{\'{A}} in the latin-2
character set.  \rtf2latex2e{} uses \textit{rtfencoding-cp1250} to translate this character to
\texttt{Aacute}.  Finally, \texttt{Aacute} is translated into the
{\LaTeX} representation \verb#\'{A}# using the
file \textit{latex-encoding}.\footnote{Although now it is just
translated to the Unicode value \texttt{0xE1}.  Anyhow, you get the idea.}
This two-step character mapping allows for
easy addition of support for additional character sets such as latin-5
(Turkish) or cp1251 (Russian).

\subsection{Test files \label{test}}
There are several test files in the \textit{test} directory of
the \rtf2latex2e distribution that demonstrate the capabilities
of the converter. You can also download a larger set of test files
to see how the program behaves.  These test files are in a tarred
gzipped archive in the same place where you downloaded the rtf2latex2e
distribution.  ``{\em RTF-test-files}'' contains several
RTF files that have been successfully tested on \rtf2latex2e.  By
success, I mean that \rtf2latex2e processes the RTF file without any
problems (except maybe giving a few warnings) and produces a ``.tex''
file that is \LaTeXe-able!!  It does not mean that the {\LaTeXe} output
file will look exactly the same as the RTF input file.  In fact, most
of the time, it will not.  Some features like I do not care to
convert, others like Unicode support will be implemented in future
versions.

\section{Acknowledgements}
I would not even have attempted this thing had it not been for Paul
DuBois' very nicely designed RTF tool.  I did not have to bother with
parsing the RTF tokens and understanding it.  All I had to do was
write code to act upon the token.  Thanks, Paul, for simplifying it. 

Steve Swanson of Mackichan Software, makers of Scientific Word and
Workplace, contributed the early
equation converter code. With this ability, \rtf2latex2e has
advanced to version 1.0. Hopefully, this essential feature addition
along with \rtf2latex2e's other capabilities will make this program
the \textit{de facto} tool for converting word processor documents
to \LaTeXe.

\section{Legalese}
This program is free software; you can redistribute it and/or
modify it under the terms of the GNU General Public License
as published by the Free Software Foundation.

The initial equation converter capability was provided by Steve Swanson
from \href{http://www.mackichan.com}{Mackichan Software},
makers of Scientific Word and Workplace.  

This program is distributed in the hope that it will be useful, but
WITHOUT ANY WARRANTY; without even the implied warranty of
MERCHANTABILITY or FITNESS FOR A PARTICULAR PURPOSE. See the GNU
General Public License for more details.  If you format your hard
disk, or do anything else inconvenient, it's not my fault.

The reader part of this code is copyright Paul DuBois.  

If you make any modifications that you think makes this program
better, please send me the modifications so that I can incorporate
them in later versions.  Please do not distribute modified versions. 
I plan to keep working on this project, and anybody is welcome to
help.
  
\end{document}
