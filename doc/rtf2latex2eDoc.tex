\documentclass[11pt]{article}

\usepackage{color}
%\usepackage{times}
%\usepackage[pdfmark,colorlinks=true,urlcolor=blue]{hyperref}
\usepackage{hyperref}

\definecolor{color1}{rgb}{0.000,0.000,0.000}
\definecolor{color2}{rgb}{0.000,0.000,1.000}
\definecolor{color3}{rgb}{0.000,1.000,1.000}
\definecolor{color4}{rgb}{0.000,1.000,0.000}
\definecolor{color5}{rgb}{1.000,0.000,1.000}
\definecolor{color6}{rgb}{1.000,0.000,0.000}
\definecolor{color7}{rgb}{1.000,1.000,0.000}
\definecolor{color8}{rgb}{1.000,1.000,1.000}
\definecolor{color9}{rgb}{0.000,0.000,0.502}
\definecolor{color10}{rgb}{0.000,0.502,0.502}
\definecolor{color11}{rgb}{0.000,0.502,0.000}
\definecolor{color12}{rgb}{0.502,0.000,0.502}
\definecolor{color13}{rgb}{0.502,0.000,0.000}
\definecolor{color14}{rgb}{0.502,0.502,0.000}
\definecolor{color15}{rgb}{0.502,0.502,0.502}
\definecolor{color16}{rgb}{0.753,0.753,0.753}



\def\rtf2latex2e{{\bf rtf2\LaTeXe}}
% for links to URLs
\def\urlone#1{\mbox{\href{#1}{\tt #1}}}
\def\urltwo#1#2{\mbox{\href{#1}{\tt #2}}}


\title{\rtf2latex2e Documentation\\
version 2.0.0} 
\author{Ujwal S. Sathyam\\ Scott Prahl}
\date{18 Jan 2011}


\begin{document} \maketitle


\section{Introduction 
\label{intro}} \rtf2latex2e is an RTF$\rightarrow${\LaTeX}
converter that takes as its input RTF files produced by Microsoft Word
and comparable word processors such as Star Office and generates a
\TeX-able ``{\bf .tex}'' file.  It has the capability to handle fairly
complex RTF files containing figures, tables, and starting from
version 1.0 equations (courtesy Steve Swanson, Mackichan
Software).  \rtf2latex2e is written using standard $C$ and
should compile on any platform supporting a $C$ compiler.  It has
been tested on Mac OS X, Linux (Intel), and Windows.

\rtf2latex2e uses the generic
\urltwo{http://www.primate.wisc.edu/software/RTF/}{RTF reader}
framework by Paul DuBois.  The framework is a general purpose tool for
processing RTF files and may be configured in a well-defined manner to
allow it to be used with a variety of writers generating different
output formats.  This provides a method for generating RTF-to-XXX
translators.  Essentially, \rtf2latex2e provides the {\LaTeXe} writer
code to the RTF reader.  Paul seems to have stopped developing the
reader code, therefore Ujwal Sathyam had to adapt it quite a bit to handle the
latest version of RTF.

\section*{What you will get}
If you expect a WYSIWYG reproduction of your RTF file, you may be
disappointed.  Our main concern has been translating the essential
features of the RTF file such as characters, figures, tables, and
equations.  Visual formatting
such as ruler positions, tabs (until I figure out a good way of doing
this), paragraph indentations, and other fluff has been largely ignored.  
The translated {\LaTeXe} file will require some manual editing to put the
finishing touches.  We just want to make that task a little easier.  In
our opinion, expecting a WYSIWYG reproduction is not practical and
misses the point entirely.

\section{Installation}
Compiled binaries are available for Macintosh and Windows systems. 
Users of Unix/Linux systems will have to compile the source code,
which is a fairly straightforward process.

\subsection{Macintosh}
The Macintosh distribution includes a FAT drag-and-drop application on
which you can drop multiple RTF files.  The output {\LaTeX} files will
be created in the same directory as the input RTF files.  If Mac users
want to build their own binaries, I have included the CodeWarrior 5.3
project file.  The project uses the DropUnix application framework
(\urlone{http:/www.zenspider.com}) that is included in the distribution.
 If you have CodeWarrior 5.3, you should be able to open the project
file and issue a ``Make'' command. For drag and drop operation, the
input RTF files need to be of type `TEXT' or `RTF ' (note the space).

\subsection{Unix/Linux \label{Unix_install}}
There is a \textit{Unix} directory that contains scripts for
configuring, building, and installing \rtf2latex2e. Change to the
\textit{Unix} directory and type:\\
\begin{tabular}{ll}
&\textbf{./configure}\\
&\textbf{make}
\end{tabular}

\noindent
This will compile the sources and create a binary called
\textbf{rtf2latex2e.bin} in the parent distribution directory. You can
optionally issue a ``make clean'' command to remove the object files.

You can choose to install the binary in a convenient location.
If you want to install into the default directory
/usr/local/rtf2latex2e, you will need to become super-user at this point.
Installation is done by:\\
\begin{tabular}{ll}
&\textbf{make install (as root)}
\end{tabular}

\noindent
The default installation directory is /usr/local/rtf2latex2e.  A
symbolic link /usr/bin/rtf2latex2e is created pointing to
\$(INSTALL\_DIR)/rtf2latex2e.bin. If there is already an existing
\rtf2latex2e installation, it will rename that directory to
``rtf2latex2e.old''. If you do not have super-user privileges, you can
edit the Makefile and change the INSTALL\_DIR to somewhere in your home
directory, say \$(HOME)/rtf2latex2e. \textbf{Make sure that the
INSTALL\_DIR path ends with ``rtf2latex2e''.}

Finally, the environment variable RTF2LATEX2E\_DIR will need to be set
from within your shell.  The variable has to point to the directory into
which rtf2latex2e was installed.  You can set the variable using\\
export RTF2LATEX2E\_DIR=directory (bash) or\\ 
setenv RTF2LATEX2E\_DIR
directory (csh)\\ 
in your .bashrc or .login file, whichever is read by
your shell.

You can also optionally install the ImageMagick image manipulation package
available at \urlone{http://www.imagemagick.org/}.
If this is installed, \rtf2latex2e will use the ImageMagick
utility ``convert'' to attempt to convert embedded PNG, JPEG,and PICT
images to EPS. This support is statically built into the Mac and Windows binaries.


\subsection{Windows}
Windows users get a pre-compiled binary of \rtf2latex2e to be run from
the MS-DOS prompt. Just run it with the RTF file to be converted as the argument.

\section{Use}
On the Macintosh, drop your RTF file onto the application. The
output {\LaTeXe} file will be generated in the same folder as the input RTF file.
On Linux/Unix and DOS, you have to run the program in a shell:\\
\textbf{rtf2latex2e} $<rtfFileName>$\\
If the file name contains spaces, enclose the path in double quotes. 

\subsection{Command-line options}
No very useful command line options are supported yet.  There are the
obvious ``-h'' for help and ``-v'' for version number.  The only other
supported option is ``-t $<$output-map-file$>$'' where you can specify
an output map file other than the default ``latex-encoding''.

\subsection{The r2l-pref preference file}
\rtf2latex2e reads a preference file ``\textit{r2l-pref}'' where you can
specify various options such \textbf{ignoreRulerSettings},
\textbf{ignoreColor}, etc.  The options are self-explanatory.  There are
also some Macintosh-specific options at the end for converting embedded
PICT images to EPS.  The PICT$\rightarrow$EPS conversion routine
requires the Apple Laserwriter driver to be installed. All the options
in the ``\textit{r2l-pref}'' file are available as a
``Preferences'' menu item in the Macintosh application.  In the
Unix/Linux and Windows versions, the default ``\textit{r2l-pref}'' file
in the installation directory can be overridden by a file with the same
name in the current working directory.

\subsection{The output map file}
The default output map file is ``latex-encoding''. The output map file determines
the {\LaTeXe} representation of characters. You could use a different output
map file, e.g. ``latex-encoding.mac'' that generates characters in accordance
to the Macintosh character set. Obviously, such a LaTeX file might need to
use an appropriate ``input'' encoding package. Packages required by an
output map file are specified in the file itself, so that the packages
are always loaded when the output map file is used. For example,
the ``latex-encoding.mac'' file
specifies ``\%{\textbackslash}usepackage[applemac]\{inputenc\}''.
The percent sign is not a comment, but tells \rtf2latex2e to treat
the following string as a output map qualifier.
You can add other qualifiers such as ``\%{\textbackslash}usepackage[T1]\{fontenc\}''.

\subsection{The r2l-head file}
\rtf2latex2e also reads a file (if present) called
``\textit{r2l-head}''.  In this file, you can specify any additional
packages that you want to use in your {\LaTeXe} file, e.g. a babel
hyphenation package or a font encoding.  The contents of this file are
just copied into the preamble of the {\LaTeXe} file.  In the Unix/Linux
and Windows versions, the default ``\textit{r2l-head}'' file in the
installation directory can be overridden by a file with the same name
in the current working directory.

\subsection{The r2l-map file}
\rtf2latex2e also reads a file (if present) called
``\textit{r2l-map}''.  In this file, you can customize options for the
document class in the {\LaTeXe} file and mappings for section headings. 
You can also specify how text style is to be handled in the {\LaTeXe}
file, e.g.\thinspace``\textbackslash {textbf}'' vs.\thinspace
\{\textbackslash bf \ldots\}.  In the Unix/Linux and Windows versions,
the default ``\textit{r2l-map}'' file in the installation directory
can be overridden by a file with the same name in the current working
directory.



\section{Features} 
\rtf2latex2e is designed to convert journal articles, reports, and
letters written in Microsoft Word.  That means I would like it to
handle the following:

\begin{itemize}

\item 
\textbf{Text Style:} Some amount of stylized text like
{\color{color6} c}{\color{color2} o}{\color{color4} l}{\color{color10}
o}r, {\bf bold}, {\em italic}, {\underline {underlined}}, and relative
size like {\small small}, normal, {\LARGE big}, {\huge very big},
{\Huge and large}.  This is a little weak in older RTF files as the
older RTF spec is a little crappier than the newer one.  All other
font information is disregarded, as TeX can do better anyway.

\item 
\textbf{Figures:} \rtf2latex2e can read figures
of format PICT, WMF, PNG, and JPEG embedded into RTF files.  These are
the most common formats encountered in RTF files.  When \rtf2latex2e
encounters an embedded figure, it reads out the figure into a separate
file.  The output format of the figure is the same as the format it is
embedded in. \rtf2latex2e will then attempt to convert the image to EPS
using both internal and external conversion routines. For PNG and JPEG
images, the ImageMagick (\urlone{http://www.imagemagick.org/})
package is used to generate EPS files. PICT images on non-Mac systems
are also converted to EPS using ImageMagick, but the performance can be
quite poor. On Macs, you have the option of using either the Laserwriter
driver for EPS generation or ImageMagick. Generally, using the Laserwriter
yields better results.

The ImageMagick support is statically built into the binaries for the
Mac and Windows platforms. \rtf2latex2e for Unix and Linux will attempt
to externally call the ImageMagick routine ``convert'' if available.
You will need to install ImageMagick separately for EPS conversion on
these platforms.

\rtf2latex2e for the Macintosh also converts embedded
WMF files to PICT format.  All major implementations of {\TeX} on the
Macintosh can handle PICT images.   Most of the image
conversion code was written by \urltwo{mailto:prahl@ece.ogi.edu}{Scott
Prahl}.


\item 
{\bf {\underline {Equations:}}} The most common source of the RTF file
is Microsoft Word.  Equations in Word are created in Equation Editor
(MathType), and when saved into an RTF file, the equation is embedded
as an OLE object. \rtf2latex2e uses the free
cole library\footnote{which in the year 2000 was available
from http://arturo.directmail.org/filtersweb} to
extract the embedded equations from the OLE structured format.
The equation is then converted into {\LaTeX} format.
Only equations created by Equation Editor 3.0 (supplied along with MS Word)
have been tested. MS Word also embeds the equation as a picture for
older RTF readers.  If the native equation conversion fails, or if
the option to convert equations is disabled, \rtf2latex2e reads that
picture and outputs the equation as a picture file.
The equation converter capability was provided by Steve Swanson
from Mackichan Software (\urlone{http://www.mackichan.com}),
makers of Scientific Word and Workplace.


\item 
{\bf {\underline {Tables:}}} Yeah, it does tables!!  However, this is
the weakest link in the chain and the messiest part of the code.  This
is largely due to the fact that RTF does not have a separate `Table'
group.  It is also due to the fact that TeX likes to know in advance
the number of columns in the table, and RTF does not tell us that.  I
spent a lot of time to support tables to this extent.  A lot of the
test files have tables in them.  To get an idea of the type of tables
that \rtf2latex2e can handle, take a look at {\bf table1.rtf}, {\bf
Script.rtf}, and {\bf RTF-Spec.rtf}.  I use longtable.sty package for
generic table handling to take care of tables that span several pages.



\item
{\bf {\underline {Paragraph Style:}}} I care for alignment issues like
centering, left, and right justification.  Useful in letters.  All
other visual formatting like indentation is currently ignored until I
figure out how to translate RTF's paragraph syntax into appropriate
LateX commands/environments.


\item 
{\bf {\underline {Character mapping:}}} Character mapping is largely
complete for the most common latin scripts.  Characters are translated
by referencing character set maps and the output map file
``latex-encoding''.  The platform and locale dependent character set, eg. 
latin-2 (Eastern European), is converted to an internal
platform-independent representation by reading the appropriate
character map file, in this case cp1250.map.  For example, character
192 (hex $0xc1$) represents ``\textit{\'{A}}'' in the latin-2
character set.  \rtf2latex2e maps this character to
``\textit{Aacute}''.  This mapping is then finally translated into the
{\LaTeX} representation ``\textbackslash$'$\{A\}'' using the output
map file \textit{latex-encoding}.  This two-step character mapping allows for
easy addition of support for additional character sets such as latin-5
(Turkish).  Also, some users prefer to use the \textit{inputenc}
package to represent characters above ASCII value 127, ie.  type
``\textit{\'{A}}'' instead of ``\textbackslash$'$\{A\}''. 
\rtf2latex2e can do this automatically by using the appropriate
mapping in the latex-encoding file.  Sample latex-encoding files for cp1252 (ANSI),
cp1250 (latin-2), and applemac (Macintosh) are provided to illustrate
this approach.  To use any of these files, rename the file to
``latex-encoding''.  The appropriate ``\textbackslash
usepackage[\ldots]\{inputenc\}'' entry in the r2l-head file will cause
the inputenc package to be loaded.

\item
{\bf {\underline {Footnotes:}}} It was quite simple to add footnote
support.  I initially had trouble converting footnotes within tables,
but it works now.

\item
{\bf {\underline {Hyperlinks:}}} I have added support for translating
hyperlinks using the hyperref package.  This is still somewhat
experimental.  There is an option in the r2l-pref file to turning off
this option.


\end{itemize}

\noindent
Features I would like to support in future versions are:
\begin{itemize}
\item
Unicode: This should really get rid of the need for different
character set maps.  Word 98 on the Mac already puts out Unicode.
\item
Lists
\end{itemize}

\section{Test files \label{test}}
There are four test files in the \textit{examples} directory of
the rtf2latex2e distribution that demonstrate the capabilities
of the converter. You can also download a larger set of test files
to see how the program behaves.  These test files are in a tarred
gzipped archive in the same place where you downloaded the rtf2latex2e
distribution.  ``{\em RTF-test-files}'' contains several
RTF files that have been successfully tested on \rtf2latex2e.  By
success, I mean that \rtf2latex2e processes the RTF file without any
problems (except maybe giving a few warnings) and produces a ``.tex''
file that is \LaTeXe-able!!  It does not mean that the {\LaTeXe} output
file will look exactly the same as the RTF input file.  In fact, most
of the time, it will not.  Some features like I do not care to
convert, others like Unicode support will be implemented in future
versions.


\section{Acknowledgements}
I would not even have attempted this thing had it not been for Paul
DuBois' very nicely designed RTF tool.  I did not have to bother with
parsing the RTF tokens and understanding it.  All I had to do was
write code to act upon the token.  Thanks, Paul, for simplifying it. 
Another great help has been the DropUnix application framework by Ryan
Davis that makes porting between command-line Unix and drag-and-drop
Macintosh a matter of changing one line of code.  DropUnix itself is
based on the drag-and-drop DropShell framework by Leonard Rosenthol,
Marshall Clow, and Stephan Somogyi.

Steve Swanson of Mackichan Software, makers of Scientific Word and
Workplace (\urlone{http://www.mackichan.com}), contributed the
equation converter code. With this ability, \rtf2latex2e has
advanced to version 1.0. Hopefully, this essential feature addition
along with \rtf2latex2e's other capabilities will make this program
the \textit{de facto} tool for converting word processor documents
to \LaTeXe.

Finally, I have to thank Scott Prahl for providing constant feedback
and encouragement to keep this going.  Scott also joined me in the
development effort and contributed the image conversion code.

\section{Legalese}
This program is free software; you can redistribute it and/or
modify it under the terms of the GNU General Public License
as published by the Free Software Foundation.

The ImageMagick library and its assciated libraries carry their respective copyrights.

The JPEG{$\rightarrow$}EPS conversion routine was adapted from Thomas
Merz's jpeg2ps program with his permission. Any copyright notices
regarding jpeg2ps still apply to the adapted code within \rtf2latex2e.
Thomas Merz's homepage is \urlone{http://www.pdflib.com/}


This program is distributed in the hope that it will be useful, but
WITHOUT ANY WARRANTY; without even the implied warranty of
MERCHANTABILITY or FITNESS FOR A PARTICULAR PURPOSE. See the GNU
General Public License for more details.  If you format your hard
disk, or do anything else inconvenient, it's not my fault.

The reader part of this code is copyright Paul DuBois.  The Macintosh
DropUnix framework is by Ryan Davis, and the DropShell part of the
code by its authors.

If you make any modifications that you think makes this program
better, please send me the modifications so that I can incorporate
them in later versions.  Please do not distribute modified versions. 
I plan to keep working on this project, and anybody is welcome to
help.

    
\end{document}
